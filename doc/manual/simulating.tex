\chapter{Simulating Probabilistic Models}

ToPS provides a program, called  \textit{simulate}, which samples  sequences from a probabilistic model,  requires as parameters the length  and the number of sequences to be generated. For example, the command line below determines the generation of $10$ sequences, each  with length $1000$, using the HMM (hmm.txt), in standard output (screen). In this case, since we are using an HMM, the output consists of pairs of sequences (the second sequence of the pair  corresponding to the hidden state labels).


\begin{Verbatim}[frame=single, label={Command line}]
simulate -m cpg_island.txt -n 10 -l 1000 -h
\end{Verbatim}

The command line parameters of the \textit{simulate} program are:
\begin{itemize}
\item -m specifies the name of the file containing the model description.
\item -n specifies  the number of sequences that will be generated.
\item -l specifies the length of each sequence.
\item -h determines the generation of the symbols and the hidden state labels.
\end{itemize}

\begin{Verbatim}[frame=single, label={hmm.txt}]
model_name="HiddenMarkovModel"
state_names= ("1", "2" )
observation_symbols= ("0", "1" )
# transition probabilities
transitions = ("1" | "1": 0.9;
                            "2" | "1": 0.1;
                            "1" | "2": 0.05;
                       "2"| "2": 0.95 )
# emission probabilities
emission_probabilities = (
                         "0" | "2" : 0.95; 
                         "1" | "2" : 0.05;            
                         "0" | "1" : 0.95; 
                         "1" | "1" : 0.05)
initial_probabilities= ("1": 0.5; "2": 0.5)
\end{Verbatim}

The simulate program is not limited to the use with HMM, any probabilistic model description works as an input.



